\usepackage{listings}
\usepackage{color}

\definecolor{dkgreen}{rgb}{0,0.6,0}
\definecolor{gray}{rgb}{0.5,0.5,0.5}
\definecolor{mauve}{rgb}{0.58,0,0.82}

\lstset{frame=tb,
  language=Java,
  aboveskip=3mm,
  belowskip=3mm,
  showstringspaces=false,
  columns=flexible,
  basicstyle={\small\ttfamily},
  numbers=none,
  numberstyle=\tiny\color{gray},
  keywordstyle=\color{blue},
  commentstyle=\color{dkgreen},
  stringstyle=\color{mauve},
  breaklines=true,
  breakatwhitespace=true,
  tabsize=3
}

\lstdefinelanguage{XML}
{
  morestring=[b]",
  morestring=[s]{>}{<},
  morecomment=[s]{<?}{?>},
  stringstyle=\color{black},
  identifierstyle=\color{darkblue},
  keywordstyle=\color{cyan},
  morekeywords={xmlns,version,type}% list your attributes here
}


\section{Software Stack}

The design goals of the software stack are:
\begin{itemize}
\item Open source
\item Active community support
\item Cross-platform
\end{itemize}

\subsection{Platform}
\subsubsection{Operating System}
Official \href{https://www.olimex.com/wiki/A20-OLinuXino-LIME#How_to_generate_boot-able_SD-card_Debian_Linux_image_for_A20-OLinuXino-LIME.3F}{Olimex LIME Debian} based on Linux kernel 3.4.90+ downloaded as a SD-card image file . Custom linux systems may also be built following tutorials on the download page.

\subsubsection{Interpreter}

Package:	\href{https://www.python.org/downloads/release/python-279/}{Python 2.7}

Python is a simple, cross-platform, object-oriented programming language that is widely used across the open-source community.

\subsubsection{SPI, I2C, GPIO}
	
Package:	\href{https://pypi.python.org/pypi/pyA20Lime}{pyA20Lime}

SPI/I2C/GPIO is used to communicate with the power management IC. GPIO is used to control and monitor the Telit GE910-GNSS. GPIO is used to control the system, network, location and fault LEDs.
    
\subsection{Location}

\subsubsection{GPS utilities}

Package:	\href{https://github.com/Knio/pynmea2}{pynmea2}

GPS location will be determined using the on-board GPS/GSNS block inside the Telit GE910-GNSS module. Location strings are given in the following format:
\begin{lstlisting}
<UTC>,<latitude>,<longitude>,<hdop>,<altitude>,<fix>,<cog>,<spkm>,<spkn>,<date>,<nsat>
\end{lstlisting}

\textbf{<UTC>} \\
UTC time (hhmmss.sss) referred to GGA sentence

\textbf{<latitude>} \\
Latitude N/S (referred to GGA sentence)
	
format:
ddmm.mmmm
					
where:
dd - degrees
00..90

mm.mmmm - minutes
00.0000..59.9999

N/S: North / South 

\textbf{<longitude>} \\
Format is dddmm.mmmm E/W (referred to GGA sentence) 

format:
dddmm.mmmm

where:
ddd - degrees						
000..180

mm.mmmm - minutes				
00.0000..59.9999

E/W: East / West

\textbf{<hdop>} \\
Horizontal Diluition of Precision (referred to GGA sentence) 

\textbf{<altitude>} \\
Altitude - mean-sea-level (geoid) in meters (referred to GGA sentence)

\textbf{<fix>} \\
0 - Invalid Fix
2 - 2D fix
3 - 3D fix

\textbf{<cog>} \\
Course over Ground (degrees, True) (referred to VTG sentence) where:
format:
ddd.mm

ddd - degrees		 	 	 		
000..360

mm - minutes						
00..59


\textbf{<spkm>}\\
Speed over ground (Km/hr) (referred to VTG sentence)

\textbf{<spkn>} \\
Speed over ground (knots) (referred to VTG sentence)

\textbf{<date>} \\
Date of Fix (referred to RMC sentence)

format:
ddmmyy 
						
where:
dd - day					
01..31

mm - month					
01..12

yy - year						
00..99 - 2000 to 2099

\textbf{<nsat>}
nn - Total number of satellites in use (referred to GGA sentence)
						
00..12

For example:
\begin{lstlisting}
194118.000,2553.37745S,02804.20558E,1.0,1516.06,3,0.0,0.1,0.1,050714,07
\end{lstlisting}

pynmea2 is used to parse Telit NMEA strings into Python objects so they can integrated as MQTT updates.

\subsection{Remote telemetry}

\subsubsection{MQTT}
Package:	\href{https://eclipse.org/paho/clients/python/}{paho}\\

\subsection{GSM/GPRS Dialer}
Package: 	ppp with wvdial\\
See Appendix B for wvdial configuration file.

\subsubsection{Sub-process interface}
Package:	\href{https://docs.python.org/2/library/subprocess.html}{subprocess}\\

\subsection{SDR}
\subsubsection{SDR hardware interface}
Package:	\href{https://github.com/steve-m/librtlsdr}{rtlsdr}\\

\subsubsection{rtlsdr wrapper}
Package:	\href{https://github.com/roger-/pyrtlsdr}{pyrtlsdr}\\
The software defined radio utilities used are based on the rtlsdr drivers by Antti Palosaari, Osmocom and Eric Fry. rtl-sdr is developed by Steve Markgraf, Dimitri Stolnikov, and Hoernchen, with contributions by Kyle Keen, Christian Vogel and Harald Welte.

For a basic network scan the rtl\_power command line utility by Steve Markgraf can be used. The following command will scan from 470 - 790 MHz with a FFT bin size of 10 kHz and an integration interval of 10 seconds at a tuner gain of ~50 dB and exit after 50s and will crop the outer 20\% of the sampling bandwidth where there is a lower signal quality. It will save the data in “data.csv” in the current directory.
\begin{lstlisting}
sudo rtl_power -f 470M:790M:0.01M -i 10s -g 50 -e 50s -c 20\%  data.csv
\end{lstlisting}

A description of the rtl\_power parameters may be found in the appendices.

All functions in librtlsdr are accessible via a Pythonic interface from pyrtlsdr.
the following python code snippet is an example of accessing the rtl-sdr utilities in python :

\begin{lstlisting}
from rtlsdr import RtlSdr

sdr = RtlSdr()

# configure device
sdr.sample_rate = 2.048e6  # Hz
sdr.center_freq = 70e6 	# Hz
sdr.freq_correction = 60   # PPM
sdr.gain = 'auto'

print(sdr.read_samples(512))
\end{lstlisting}

\subsubsection{n-dimensional arrays}
Package:	NumPy

\subsubsection{Database}
Package:	SQLite

\subsubsection{SQLite interface}
Package:	sqlite3
    
    
    
    
    
    
    
    
    
    
    
    
    
    