\usepackage{listings}
\usepackage{color}

\definecolor{dkgreen}{rgb}{0,0.6,0}
\definecolor{gray}{rgb}{0.5,0.5,0.5}
\definecolor{mauve}{rgb}{0.58,0,0.82}

\lstset{frame=tb,
  language=Java,
  aboveskip=3mm,
  belowskip=3mm,
  showstringspaces=false,
  columns=flexible,
  basicstyle={\small\ttfamily},
  numbers=none,
  numberstyle=\tiny\color{gray},
  keywordstyle=\color{blue},
  commentstyle=\color{dkgreen},
  stringstyle=\color{mauve},
  breaklines=true,
  breakatwhitespace=true,
  tabsize=3
}


\section{Software Stack}

The design goals of the software stack are:
\begin{itemize}
\item Open source
\item Active community support
\item Cross-platform
\end{itemize}

\subsection{Platform}
\subsubsection{Operating System}
Official \href{https://www.olimex.com/wiki/A20-OLinuXino-LIME#How_to_generate_boot-able_SD-card_Debian_Linux_image_for_A20-OLinuXino-LIME.3F}{Olimex LIME} Debian based on Linux kernel 3.4.90+ downloaded as a SD-card image file. Custom Linux systems may also be built following tutorials on the download page.

\subsubsection{Interpreter}

Package:	\href{https://www.python.org/downloads/release/python-279/}{Python 2.7}

Python is a simple, cross-platform, object-oriented programming language that is widely used across the open-source community.

\subsubsection{SPI, I2C, GPIO}
	
Package:	\href{https://pypi.python.org/pypi/pyA20Lime}{pyA20Lime}

SPI/I2C/GPIO is used to communicate with the power management IC. GPIO is used to control and monitor the Telit GE910-GNSS. GPIO is used to control the system, network, location and fault LEDs.

\subsection{Location}

\subsubsection{GPS utilities}

Package:	\href{https://github.com/Knio/pynmea2}{pynmea2}

The SSM's location will be determined using the on-board \href{http://egnos-portal.gsa.europa.eu/discover-egnos/about-egnos/what-gnss}{GNSS} block inside the Telit GE910-GNSS module. pynmea2 is used to parse \href{http://www.nmea.org/content/nmea_standards/nmea_0183_v_410.asp}{NMEA0183} formatted strings strings into Python objects so they can be integrated as MQTT updates. See the Appendix for the parsed data format and structure.

\subsection{Remote telemetry}

\subsubsection{MQTT}

\href{http://mqtt.org/}{MQTT} is a connectivity protocol for Machine to Machine communication. It is a lightweight publish/subscribe based messaging transport protocol. It is used for connections with remote locations over unreliable networks and where data cost is at a premium. It is based on an efficient broker based distribution system to easily allow  information flow between one or many receivers. 

Package:	\href{https://eclipse.org/paho/clients/python/}{paho}\\
The Paho project is an open-source client implementations of MQTT and MQTT-SN messaging protocols. It is implemented in python using the paho python client.



\subsection{GSM/GPRS Dialer}
Package: 	ppp with wvdial\\
See Appendix B for wvdial configuration file.

\subsubsection{Sub-process interface}
Package:	\href{https://docs.python.org/2/library/subprocess.html}{subprocess}\\

\subsection{SDR}
\subsubsection{SDR hardware interface}
Package:	\href{http://sdr.osmocom.org/trac/wiki/rtl-sdr}{rtl-sdr}

A \href{http://sdr.osmocom.org/trac/wiki/rtl-sdr#Buildingthesoftware}{community built} software defined radio interface is used to capture I/Q sampled radio data from a \href{http://www.rtl-sdr.com/buy-rtl-sdr-dvb-t-dongles/}{ USB DVB-T TV tuner device} with a Realtek RTL2832U chipset. \href{http://kmkeen.com/rtl-power/}{RTL-power} is used to capture, process and combine spectrum power plots that exceed the instantaneous bandwidth of the tuner module.

\subsubsection{rtlsdr wrapper}
Package:	\href{https://github.com/roger-/pyrtlsdr}{pyrtlsdr}\\
The software defined radio utilities used are based on the rtlsdr drivers by Antti Palosaari, Osmocom and Eric Fry. rtl-sdr is developed by Steve Markgraf, Dimitri Stolnikov, and Hoernchen, with contributions by Kyle Keen, Christian Vogel and Harald Welte.

For a basic network scan the rtl\_power command line utility by Steve Markgraf can be used. The following command will scan from 470 - 790 MHz with a FFT bin size of 10 kHz and an integration interval of 10 seconds at a tuner gain of ~50 dB and exit after 50s and will crop the outer 20\% of the sampling bandwidth where there is a lower signal quality. It will save the data in “data.csv” in the current directory.
\begin{lstlisting}
sudo rtl_power -f 470M:790M:0.01M -i 10s -g 50 -e 50s -c 20\%  data.csv
\end{lstlisting}

A description of the rtl\_power parameters may be found in the appendices.

All functions in librtlsdr are accessible via a Pythonic interface from pyrtlsdr.
the following python code snippet is an example of accessing the rtl-sdr utilities in python :

\begin{lstlisting}
from rtlsdr import RtlSdr

sdr = RtlSdr()

# configure device
sdr.sample_rate = 2.048e6  # Hz
sdr.center_freq = 70e6 	# Hz
sdr.freq_correction = 60   # PPM
sdr.gain = 'auto'

print(sdr.read_samples(512))
\end{lstlisting}

\subsubsection{n-dimensional arrays}
Package:	NumPy

\subsubsection{Database}
Package:	SQLite

\subsubsection{SQLite interface}
Package:	sqlite3
    
    
    
    
    
    
    
    
    
    
    
    
    
    
    
    
    
    
    
    
    
    
    
    
    
    
    