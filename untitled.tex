\section{Software Stack}

The design goals of the software stack are:
\begin{itemize}
\item Open source
\item Active community support
\item Cross-platform
\end{itemize}

\subsection{Platform}
\subsubsection{Operating System}
Official \href{https://www.olimex.com/wiki/A20-OLinuXino-LIME#How_to_generate_boot-able_SD-card_Debian_Linux_image_for_A20-OLinuXino-LIME.3F}{Olimex LIME Debian} based on Linux kernel 3.4.90+ downloaded as a SD-card image file . Custom linux systems may also be built following tutorials on the download page.

\subsubsection{Interpreter}

Package:	\href{https://www.python.org/downloads/release/python-279/}{Python 2.7}

Python is a simple, cross-platform, object-oriented programming language that is widely used across the open-source community.

\subsubsection{SPI, I2C, GPIO}
	
Package:	\href{https://pypi.python.org/pypi/pyA20Lime}{pyA20Lime}

SPI/I2C/GPIO is used to communicate with the power management IC. GPIO is used to control and monitor the Telit GE910-GNSS. GPIO is used to control the system, network, location and fault LEDs.
    
\subsection{Location}

\subsubsection{GPS utilities}

Package:	\href{https://github.com/Knio/pynmea2}{pynmea2}

GPS location will be determined using the on-board GPS/GSNS block inside the Telit GE910-GNSS module. Location strings are given in the following format:
\begin{quote}
<UTC>,<latitude>,<longitude>,<hdop>,<altitude>,<fix>,<cog>,<spkm>,<spkn>,<date>,<nsat>
\end{quote}


    \begin{tabular}{| l | l |}
        <UTC> & UTC time (hhmmss.sss) referred to GGA sentence \\ \hline
        <latitude> & Latitude N/S (referred to GGA sentence)format:ddmm.mmmm	where:dd - degrees00..90mm.mmmm - minutes00.0000..59.9999N/S: North / South \\ \hline
        <longitude> & Format is dddmm.mmmm E/W (referred to GGA sentence) format:dddmm.mmmmwhere:ddd - degrees	000..180mm.mmmm - minutes	00.0000..59.9999E/W: East / West \\ \hline
        <hdop> & Horizontal Diluition of Precision (referred to GGA sentence) \\ \hline
        <altitude> & Altitude - mean-sea-level (geoid) in meters (referred to GGA sentence) \\ \hline
        <fix> & 0 - Invalid Fix2 - 2D fix3 - 3D fix \\ \hline
        <cog> & Course over Ground (degrees, True) (referred to VTG sentence) where:

format:ddd.mmddd - degrees000..360mm - minutes00..59 \\ \hline
        <spkm> & Speed over ground (Km/hr) (referred to VTG sentence) \\ \hline
        <spkn> & Speed over ground (knots) (referred to VTG sentence) \\ \hline
        <date> & Date of Fix (referred to RMC sentence) format:ddmmyy where:dd - day01..31mm - month01..12yy - year00..99 - 2000 to 2099 \\ \hline
        <nsat> & nn - Total number of satellites in use (referred to GGA sentence)00..12 \\ \hline
         &  \\ \hline
    \end{tabular} 

    
    
    
    
\begin{tabular}{ | l | l | l | p{5cm} |}
   \hline
   Day & Min Temp & Max Temp & Summary \\ \hline
   Monday & 11C & 22C & A clear day with lots of sunshine.  
   However, the strong breeze will bring down the temperatures. \\ \hline
   Tuesday & 9C & 19C & Cloudy with rain, across many northern regions. Clear spells 
   across most of Scotland and Northern Ireland, 
   but rain reaching the far northwest. \\ \hline
   Wednesday & 10C & 21C & Rain will still linger for the morning. 
   Conditions will improve by early afternoon and continue 
   throughout the evening. \\
   \hline
   \end{tabular}
    
    